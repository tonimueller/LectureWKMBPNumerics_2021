%% LyX 2.3.4.4 created this file.  For more info, see http://www.lyx.org/.
%% Do not edit unless you really know what you are doing.
\documentclass[12pt,german,journal=mamobx,manuscript=article,maxauthors=15,biblabel=plain]{article}
\usepackage[T1]{fontenc}
\usepackage[latin9]{inputenc}
\usepackage[a4paper]{geometry}
\geometry{verbose,tmargin=1in,bmargin=1in,lmargin=1in,rmargin=1in,headheight=0.5cm,headsep=0.5cm,footskip=0.5cm}
\setlength{\parskip}{\smallskipamount}
\setlength{\parindent}{0pt}
\usepackage{amsmath}
\usepackage{url}
\makeatletter
%%%%%%%%%%%%%%%%%%%%%%%%%%%%%% User specified LaTeX commands.
\usepackage{babel}

\makeatother

\usepackage{babel}
\begin{document}

\section*{\"Ubung 4}
BBB at 20.05.2021: \\
Teilnehmende mit ZIH-Login:\\
\url{https://selfservice.zih.tu-dresden.de/l/link.php?m=111130&p=32a58878}

Teilnehmende ohne Hochschullogin:\\
\url{https://selfservice.zih.tu-dresden.de/link.php?m=111130&p=8ea97381}

\subsubsection*{Adsorption von Monomeren}

Wir betrachten ein System mit Teilchen (Monomere) ohne paarweise Wechselwirkungen.
Die Simulationsbox sei in zwei Raumrichtungen periodisch, in einer
nicht-periodisch. Diese Wand hat eine attraktive Wechselwirkung mit
den Monomeren, wobei die Wechselwirkungsst\"arke durch die Energie pro
Monomer-Wand-Kontakt $\delta$ gegeben ist. Sie k\"onnen das Modul lattice\_gas.py
benutzen.
\begin{enumerate}
\item Erstellen Sie eine Startkonformation mit $256$ einzelnen Monomeren
in einer kubischen Box mit $32^{3}$ Gitterpl\"atzen.
\item Implementieren Sie absorbierende W\"ande um diese Monomere an zwei
gegen\"uberliegenden W\"anden mit verschiedenen Adsorptionsenergien $\delta\in[0,10]$
zu simulieren. Nutzen Sie dazu den Metropolis Algorithmus. Messen
Sie die Anzahl adsorbierter Monomere $N_{\text{ads}}$.
\item Tragen Sie die Adsorptionsisotherme $N_{\text{ads}}/N(\delta)$ auf
und vergleichen Sie mit der analytischen Vorhersage der Boltzmann
Statistik: 
\[
\frac{N_{\text{ads}}}{N}(\delta)=\frac{\exp(\delta/k_{\text{B}}T)}{\exp(\delta/k_{\text{B}}T)+V_{0}/V_{\text{ads}}}
\]
Dabei ist $N$ die Gesamtanzahl der Teilchen in der Box mit dem Gesamtvolumen
$V$. $V_{\text{ads}}$ bezeichnet das Volumen, das f\"ur die Adsorption
zur Verf\"ugung steht, $V_{0}$ das Volumen, das nicht f\"ur die Adsorption
zur Verf\"ugung steht.
\item Leiten Sie diesen Ausdruck her. Nehmen Sie als Ausgangspunkt die Entropie
des idealen Gases $S=S(T)-k_{\text{B}}N\ln(V/v_{0}N)$ an, wobei $v_{0}$
das Eigenvolumen eines Monomeres ist.
\end{enumerate}
\subsubsection*{Polymerb\"ursten}
\begin{enumerate}
\item Nutzen Sie die das BFM zur Erstellung von Polymerketten mit ausgeschlossenem
Volumen, deren erstes Monomer an der (nicht periodischen) Grundfl\"ache
der Simulationsbox verankert ist, als Modellsystem f\"ur eine Polymerb\"urste.
Erzeugen Sie Systeme mit $8x8$ Ketten aus 32 Monomeren bei Ankerpunktdichten
$\sigma$ 1/9, 1/25, 1/256 sowie eine einzelne geankerte Kette.
\item Equilibrieren Sie die B\"ursten und bestimmen Sie die mittlere H\"ohe
$H$. Vergleichen Sie die Abh\"angigkeit $H\sim\sigma$ mit der theoretischen
Vorhersage.
\end{enumerate}

\end{document}
