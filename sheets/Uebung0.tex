%% LyX 2.3.4.4 created this file.  For more info, see http://www.lyx.org/.
%% Do not edit unless you really know what you are doing.
\documentclass[12pt,german,journal=mamobx,manuscript=article,maxauthors=15,biblabel=plain,pagestyle=plain]{article}
\usepackage[T1]{fontenc}
\usepackage[latin9]{inputenc}
\usepackage[a4paper]{geometry}
\geometry{verbose,tmargin=1in,bmargin=1in,lmargin=1in,rmargin=1in,headheight=0.5cm,headsep=0.5cm,footskip=0.5cm}
\setlength{\parskip}{\smallskipamount}
\setlength{\parindent}{0pt}
\usepackage{url}
\usepackage{amsmath}
\usepackage{amsthm}
\usepackage{amssymb}

\makeatletter
%%%%%%%%%%%%%%%%%%%%%%%%%%%%%% Textclass specific LaTeX commands.
\numberwithin{equation}{section}
\numberwithin{figure}{section}

\makeatother

\usepackage{babel}
\begin{document}

\section*{\"Ubung 0}

Dieses \"Ubungsblatt dient der Einf\"uhrung in die Software
jupyter-notebook mit Python 3 als Skriptsprache, welche w\"ahrend der 
Online-Meetings \"uber MC-Simluation benutzt wird. 

\subsection*{Informationen}
\begin{itemize}
    \item Vorlesender und \"Ubungsleiter Neuronale Netze: Marco Werner (werner-marco@ipfdd.de)
    \item \"Ubungsleiter Monte-Carlo Simulationen: Toni M\"uller (mueller-toni@ipfdd.de)
    \item Vorlesender Molekulardynamik Simulationen: Holger Merlitz (merlitz@ipfdd.de)
    \item \"Ubungsleiter Molekulardynamik Simulationen: Markus Koch (koch-markus@ipfdd.de)
    \item Zugang Materialien (Name: wkmbp, PW: ss2021): \url{https://www.ipfdd.de/de/scmbp/soft-condensed-matter-and-biological-physics/lecture-materials/}
    \item Jitsi-Raum f\"ur die \"Ubung : \url{https://jitsi.tu-dresden.de/UebungNumerikWKMBP}
    \item Zur Nutzung von Jitsi gibt es eine ausf\"uhrliche Seite der TU Dresden:
    \url{https://tu-dresden.de/tu-dresden/organisation/zentrale-universitaetsverwaltung/dezernat-3-zentrale-angelegenheiten/sg-3-5-informationssicherheit/tud-cert/videokonferenzservice}
\end{itemize}

\subsection*{Python und jupyter-notebook}

F\"ur diese \"Ubung verwenden wir die Skriptsprache python. Zum einfacheren
Gebrauch und f\"ur die unkomplizierte Darstellung \"uber die Bildschirmfreigabe
wird jupyter-notebook verwendet. Sollten Sie auf Ihrem Arbeitsger\"at
noch keine Python-Distribution und jupyter installiert haben, k\"onnen
Sie mit Anaconda schnell und einfach Ihre Arbeitsumgebung aufsetzen: 
\begin{enumerate}
\item Auf \url{www.anaconda.com/distribution/} finden Sie Anaconda zum
herunterladen f\"ur Windows, macOS und Linux
\item Installieren Sie die Python 3.7 Version nach den Anweisungen des Installers.
\item Testen Sie ihre Installation, indem Sie in dem Ordner, in dem Sie
die \"Ubungen speichern wollen, in einer Shell, der Powershell, der
ZSH, ... den Befehl jupyter-notebook(.exe unter Windows) eingeben. 
\item In Ihrem Standardbrowser sollte sich die jupyter Home Page \"offnen.
Klicken Sie auf der rechten Seite auf New/Neu und w\"ahlen Sie Notebook:
Python 3 aus.
\end{enumerate}
Lassen Sie sich ``Hallo Welt'' ausgeben. Eine detailierte Anleitung
bis zu diesem Schritt finden sie hier: \url{https://realpython.com/jupyter-notebook-introduction/}

Python ist eine objektorientierte Sprache. Darum werden zu erledigende
Aufgaben in Klassen implementiert, die eigene Variablen (Member) und
eigene (private) Funktionen enthalten. Dieses Prinzip wird in der
folgenden \"Ubung an einem einfachen Beispiel angewendet und wichtige
Bibliotheken werden vorgestellt und benutzt.

\subsection*{Der Calculator}

Schreiben Sie eine Python Klasse mit dem Name ``calculator'', mit
der Sie zwei Zahlen $a$ und $b$ addieren, substrahieren und multiplizieren
k\"onnen. Au{\ss}erdem soll die Funktion $f(x)=a*x+b$ f\"ur $x\in\mathbb{N},$
$x<20$ in eine Datei ausgeben werden, sowie eine solche Datei eingelesen
und die Werte von a und b \"uber einen Fit bestimmt werden k\"onnen. Benutzen
Sie die Bibliotheken numpy, scipy, matplotlib und pandas, und
machen sich mit den unten aufgef\"uhrten Objekten und Funktionen vertraut.

N\"utzliche Funktionen und Objekte:
\begin{itemize}
\item Am Anfang des notebooks (Interaktive Umgebung f\"ur numpy und matplotlib):\textbf{
\%pylab notebook}
\item numpy.array, pandas.DataFrame
\item matplotlib.pyplot.figure und matplotlib.pyplot.plot
\item numpy.loadtxt oder pandas.read\_csv /read\_table
\item numpy.savetxt oder pandas.DataFrame.to\_csv
\item scipy.optimize.curve\_fit
\end{itemize}
% \subsection*{Zusatzaufgabe: Numerische Intergration}
% Schreiben Sie eine Klasse ``integrator''. Implementieren Sie darin 
% numerische Integrationsmethoden zur Berechnung eines Integrals einer beliebigen Funktion 
% durch 
% \begin{enumerate}
%     \item die Approximation der Funktion durch Rechtecke (Stichwort Riemann-Integral)
%     \item die Approximation der Funktion durch Trapeze ( Trapez-Regel )
%     \item die Realisierung der Simpson-Regel.
% \end{enumerate}
% Testen Sie das Programm mit einer analytisch berechenbaren Funktion. 


\subsection*{Monte-Carlo Integration}

Nutzen Sie jetzt einen vorimplementierten Zufallszahlengenerator,
zum Beispiel numpy.random.random.
\begin{enumerate}
\item Verwenden Sie Zufallszahlen um die Kreiszahl $\pi$ zu berechnen.
Betrachten Sie ihre Analyse als Zufallsexperiment und sch\"atzen Sie
den Fehler ihrer Auswertung ab. Mit welcher maximalen Genauigkeit
k\"onnen Sie die Kreiszahl $\pi$ berechnen? 
\item Betrachten Sie die Funktion $f(x,y)=x^{2}-y^{3}+xy^{2}$. Berechnen
Sie das Volumen, das von dieser Funktion und der $xy$-Ebene f\"ur Koordinaten
aus den Intervallen $x\in[-1,1]$, $y\in[-1,1]$ eingeschlossen wird.
Stellen Sie das Ergebnis graphisch dar.
\item Beurteilen Sie die Effizienz dieses Integrations-Verfahrens im Vergleich
zu einer gleichm\"a{\ss}igen Rasterung des Raumes. Was sind die Vor- bzw.
Nachteile der Monte-Carlo Integration (hochdimensionaler Phasenraum,
periodische Funktionen, ...)? 
\end{enumerate}
N\"utzliche Funktionen:
\begin{itemize}
\item numpy.meshgrid, matplotlib.pyplot.plot\_surface
\item numpy.random.random\_sample
\end{itemize}


\end{document}
