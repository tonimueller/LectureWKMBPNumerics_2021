%% LyX 2.3.4.4 created this file.  For more info, see http://www.lyx.org/.
%% Do not edit unless you really know what you are doing.
\documentclass[12pt,german,journal=mamobx,manuscript=article,maxauthors=15,biblabel=plain]{article}
\usepackage[T1]{fontenc}
\usepackage[latin9]{inputenc}
\usepackage[a4paper]{geometry}
\geometry{verbose,tmargin=1in,bmargin=1in,lmargin=1in,rmargin=1in,headheight=0.5cm,headsep=0.5cm,footskip=0.5cm}
\setlength{\parskip}{\smallskipamount}
\setlength{\parindent}{0pt}
\usepackage{amsmath}
\usepackage{url}

\makeatletter
%%%%%%%%%%%%%%%%%%%%%%%%%%%%%% User specified LaTeX commands.
\usepackage{babel}

\makeatother

\usepackage{babel}
\begin{document}

\section*{\"Ubung 2}
BBB at 06.05.2021: \\
Teilnehmende mit ZIH-Login:\\
\url{https://selfservice.zih.tu-dresden.de/l/link.php?m=111130&p=32a58878} \\
Teilnehmende ohne Hochschullogin:\\
\url{https://selfservice.zih.tu-dresden.de/link.php?m=111130&p=8ea97381}


\subsection*{Diffusion und Zufallspfade (random walks)}

Schreiben Sie ein Programm, mit dem Sie Zufallspfade in 2D erzeugen
k\"onnen. Erstellen Sie Funktionen f\"ur die Berechnung des End-zu-End
Abstands $R_{\text{e}}$ und des Gyrationsradius $R_{\text{g}}$.
Nehmen Sie an, dass sich der L\"aufer bei jedem Schritt um eine Einheit
in genau eine der beiden Raumrichtungen bewegt.

\subsubsection*{Simple Sampling mit Zufallspfaden}
\begin{enumerate}
\item Erzeugen Sie 1000 Zufallspfade aus 512 Schritten und bestimmen Sie
die H\"aufigkeitsverteilung der Endpositionen der Zufallspfade. Vergleichen
Sie die entstehende Verteilung mit den erwarteten Verteilungen in
1D, sowie radial.
\item Berechnen Sie die mittleren End-zu-End Abst\"ande $R_{\text{e}}$ und
Gyrationsradien $R_{\text{g}}$ einer Serie an Zufallspfaden verschiedener
L\"ange (etwa $2^{4}$-$2^{11}$), um den Zusammenhang $R^{2}=b^{2}N$
zu \"uberpr\"ufen. Welche Werte erhalten Sie jeweils f\"ur die Bindungsl\"ange
$b$?
\item Diskutieren Sie die Genauigkeit/Fehlergr\"o{\ss}e ihrer Simulationen.
\end{enumerate}

\subsubsection*{Importance Sampling mit Zufallspfaden}

Erweitern Sie Ihr Programm um eine Funktion, die einen bestehenden
Zufallspfad durch Verschiebung der Endpunkte der Schritte ver\"andert,
sodass alle ver\"anderterten Schritte nach der Verschiebung wieder 
eine L\"ange von eins haben (keine diagonalen Kanten oder 
Schrittlaengen von gr\"o{\ss}er eins).
\begin{enumerate}
\item Bestimmen Sie die Autokorrelationsfunktion des End-zu-End Vektors
\begin{align*}
c(\Delta t)=\left\langle \overrightarrow{R_{\text{e}}}(t_{0})\cdot\overrightarrow{R_{\text{e}}}(t_{0}+\Delta t)\right\rangle /\left\langle \vec{R_{\text{e}}^{2}}\right\rangle 
\end{align*}
einer Kette aus 64 Bindungsvektoren, die f\"ur eine gro{\ss}e Zahl an verschiedenen
Systemen ($\ge1000$) als Funktion des Zeitintervalls $\Delta t$
gemittelt wird. Die Relaxationszeit, $\tau$, einer Polymerkette ist (nach dem Rouse-Modell)
\begin{align*}
    \tau  \approx \tau_0  N^2
\end{align*}
mit der Monomerrelaxationszeit $\tau_0$ (die Zeit die ben\"otigt wird damit sich ein Monomer 
um seine eigene Ausdehnung verschiebt). 
Wie gro{\ss} sollte die Simulationszeit gew\"ahlt werden  und f\"ur welche $\Delta t$
erh\"alt man statistisch unkorrelierte Konformationen? 
\item Wiederholen Sie die Berechnungen von $R_{\text{e}}$ und $R_{\text{g}}$
und vergleichen Sie die Effizienz.
\end{enumerate}
Zusatz: Implementieren Sie eine Kontroll-Funktion, die den aktuellen
Pfad auf die Einhaltung der Regeln des Zufallspfades \"uberpr\"uft (sanity
check).
\end{document}
