%% LyX 2.3.4.4 created this file.  For more info, see http://www.lyx.org/.
%% Do not edit unless you really know what you are doing.
\documentclass[12pt,german,journal=mamobx,manuscript=article,maxauthors=15,biblabel=plain]{article}
\usepackage[T1]{fontenc}
\usepackage[latin9]{inputenc}
\usepackage[a4paper]{geometry}
\geometry{verbose,tmargin=1in,bmargin=1in,lmargin=1in,rmargin=1in,headheight=0.5cm,headsep=0.5cm,footskip=0.5cm}
\setlength{\parskip}{\smallskipamount}
\setlength{\parindent}{0pt}
\usepackage{amsmath}
\usepackage{amssymb}
\usepackage{graphicx}
\usepackage{url}
\makeatletter
%%%%%%%%%%%%%%%%%%%%%%%%%%%%%% User specified LaTeX commands.
\usepackage{babel}

\makeatother

\usepackage{babel}
\begin{document}

\section*{\"Ubung 3}
BBB at 20.05.2021: \\
Teilnehmende mit ZIH-Login: \\
\url{https://selfservice.zih.tu-dresden.de/l/link.php?m=111131&p=f7e911f3}

Teilnehmende ohne Hochschul-Login: \\
\url{https://selfservice.zih.tu-dresden.de/link.php?m=111131&p=7f14e7bb}

\subsubsection*{Das Bindungs-Fluktuations Modell (Bond Fluctuation Model)}

Mit diesem \"Ubungsblatt verwenden Sie zum ersten Mal das sogenannte
``Bindungs-Fluktuations-Modell'' (BFM), das in der Polymerphysik
 zur Simulation von Schmelzen und L\"osungen verwendet werden kann. Die
tats\"achliche Struktur (Valenzwinkel, Rotationsbarrieren) eines amorphen
Polymers kann f\"ur eine ausreichende Zahl an Monomeren vernachl\"assigt
werden, so dass sich die statistischen Segmente bis auf das ausgeschlossene
Volumen hin frei zueinander einstellen k\"onnen:

\begin{center}
\includegraphics[width=0.9\textwidth]{Abb_3_1} 
\par\end{center}

Die Bindungsvektoren werden auf die Menge $B$

\[
B=P\pm\left(\begin{array}{c}
2\\
0\\
0
\end{array}\right)\bigcup P\pm\left(\begin{array}{c}
2\\
1\\
0
\end{array}\right)\bigcup P\pm\left(\begin{array}{c}
2\\
1\\
1
\end{array}\right)\bigcup P\pm\left(\begin{array}{c}
2\\
2\\
1
\end{array}\right)\bigcup P\pm\left(\begin{array}{c}
3\\
0\\
0
\end{array}\right)\bigcup P\pm\left(\begin{array}{c}
3\\
1\\
0
\end{array}\right)
\]

aus 108 Bindungsvektoren zwischen den Monomeren eingeschr\"ankt, um
eine \"Uberkreuzung der Ketten bei einer Bewegung eines Monomers zu
verhindern. $P\pm$ bezeichnet dabei die Menge aller m\"oglichen Permutationen
und Vorzeichenkombinationen der Vektorkoordinaten. Auf diese Weise
sind 87 verschiedene Winkeleinstellungen zwischen den Segmenten und
f\"unf Bindungsl\"angen ($2$, $\sqrt{5}$, $\sqrt{6}$, 3 und $\sqrt{10}$)
m\"oglich. Das BFM ist damit weitaus flexibler als das bisher eingesetzte
Gittermodell (\"Ubung 2) zur Simulation von Makromolek\"ulen.

Die Bewegung eines Monomers wird folgenderma{\ss}en realisiert: Zun\"achst
werden statistisch ein Monomer und eine der 6 Bewegungsrichtungen
ausgew\"ahlt. Anschlie{\ss}end wird \"uberpr\"uft, ob f\"ur die neue Position
des Monomers die Bindungsvektoren zu den Kettennachbarn in $B$ enthalten
sind und ob die 4 Gitterpl\"atze in Bewegungsrichtung noch frei sind
(im Falle energetischer Wechselwirkungen w\"urde noch der Metropolis-Algorithmus
angewandt). Nur wenn diese Kriterien alle erf\"ullt sind, wird die Bewegung
durchgef\"uhrt. Obige Abbildung zeigt einen Sprungversuch in positive
y-Richtung, bei dem alle Voraussetzungen erf\"ullt sind.

\newpage{}

\subsubsection*{Umsetzung des Bindungs-Fluktuations Modell}

Zur Implementierung des BFM in einer objektorientierten Programmiersprache
erstellen wir zun\"achst einzelne Programmteile, mit denen die einzelenen
Bestandteile des Modells umgesetzt werden. Dazu nutzen wir eine Klasse,
die die Positionen der Monomere, ihre Bindungen und ihre Attribute
bereitstellt. Das Bindungsvektorset wird ebenfalls als separate Klasse
implementiert. Das ausgeschlossene Volumen, also das Verbot, dass
ein Gitterplatz von mehreren BFM Einheiten belegt werden kann, wird
\"uber ein Gitter und entsprechende Zugriffsfunktionen darauf in einer
eigenen Klasse angelegt. Mit diesen drei Programmbausteinen wird ein
Simulationsobjekt erstellt, mit dem eine Monte-Carlo Bewegung auf
Richtigkeit \"uberpr\"uft und gegebenenfalls ausgef\"uhrt wird. Eine m\"ogliche
Realisierung finden Sie in der Datei bfm\_full.py als Python Modul.
\begin{enumerate}
\item Nutzen Sie das gegebene Programm oder eines Ihrer Wahl zur Erstellung
einer einzelnen Polymerkette mit ausgeschlossenen Volumen und $N=32$ Monomeren.
\item Implementieren Sie die Berechnung des End-zu-End Abstands $R_{\text{ee}}$
und des Gyrationsradius $R_{\text{g}}$.
\item F\"uhren Sie eine gro{\ss}e Anzahl an Simulationen durch ($\approx1000$)
und berechnen Sie damit die Autokorrelationsfunktion des End-zu-End
Vektors $\vec{R}_{\text{ee}}$.
\end{enumerate}

\subsubsection*{Polymer in eingeschr\"ankter Geometrie}
\begin{enumerate}
\item Erstellen Sie mit einem geeigneten Programm eine einzelne Polymerkette
mit und ohne ausgeschlossenem Volumem mit $N=64$ in einem Spalt der
Breite $D_{S}$ und einer Pore mit quadratischem Querschnitt der Fl\"ache
$D_{P}^{2}$ mit verschiedenen Abmessungen ($64\ge b_{\text{xy}}\ge4$).
Beobachten Sie den \"Ubergang von ungest\"orten Konformationen zu quasi
zwei- bzw. ein-dimensionalen Konformationen anhand der Ver\"anderung
des End-zu-End Abstandes und des Gyrationsradius. (Hinweis: Diese Messgr\"ossen k\"onnen 
in verschiedenen Richtungskomponenten definiert und ausgewertet werden.)
\item Berechnen sie die Paarkorrelationsfunktion $g(r)$ (radiale Verteilungsfunktion):
\[
g(r)=\frac{V}{N^{2}}\langle\sum_{j\neq i}^{N}\delta(r-|R_{\text{i}}-R_{\text{j}}|)\rangle
\]
mit der Teilchenzahl $N$ und den Monomerpositionen $\vec{R}_{\text{i,j}}$.
Welche Informationen k\"onnen Sie $g(r)$ entnehmen? Welche Gr\"o{\ss}en lassen
sich daraus ableiten?
\end{enumerate}

\end{document}
