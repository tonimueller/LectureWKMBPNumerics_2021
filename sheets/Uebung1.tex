%% LyX 2.3.4.4 created this file.  For more info, see http://www.lyx.org/.
%% Do not edit unless you really know what you are doing.
\documentclass[12pt,ngerman,journal=mamobx,manuscript=article,maxauthors=15,biblabel=plain]{article}
\usepackage[T1]{fontenc}
\usepackage[latin1]{inputenc}
\usepackage[a4paper]{geometry}
\geometry{verbose,tmargin=1in,bmargin=1in,lmargin=1in,rmargin=1in,headheight=0.5cm,headsep=0.5cm,footskip=0.5cm}
\pagestyle{empty}
\setlength{\parskip}{\smallskipamount}
\setlength{\parindent}{0pt}
\usepackage{amsmath}
\usepackage{url}
\makeatletter
%%%%%%%%%%%%%%%%%%%%%%%%%%%%%% User specified LaTeX commands.
\usepackage[ngerman]{babel}
\usepackage{babel}

\makeatother

\usepackage{babel}
\begin{document}

\section*{\"Ubung 1}
BBB at 29.04.2021: \\
Teilnehmende mit ZIH-Login:\\
\url{https://selfservice.zih.tu-dresden.de/l/link.php?m=111129&p=30b9b3d1}

Teilnehmende ohne Hochschullogin:\\
\url{https://selfservice.zih.tu-dresden.de/link.php?m=111129&p=1e8fc6be}

\subsection*{Das Ising Modell}

Schreiben Sie ein Programm, dass zuf\"allige Spin-Konformationen $\pm1$
auf einem zweidmensionalen, periodischen Gitter der Gr\"o{\ss}e $L\times L$
erstellt. Auf diesem Gitter soll das Ising Modell mit dem Hamiltonian

\[
\mathcal{H}=-\frac{1}{2}\sum_{i,j}J_{\text{ij}}S_{\text{i}}S_{\text{j}}-B\sum_{i}S_{\text{\text{i}}}
\]

implementiert werden, wobei $i$ \"uber alle Spins im System und $j$
\"uber die jeweiligen n\"achsten Nachbarn im System geht.
\begin{enumerate}
\item Schreiben Sie eine Funktion zur Berechnung der Magnetisierung $M.$
\item Schreiben Sie eine Funktion zur Berechnung der Gesamtenergie $E$.
Beachten Sie die periodische Randbedingungen.
\end{enumerate}

\subsubsection*{Simple Sampling}
\begin{enumerate}
\item Nutzen Sie Ihr Programm, um eine gro{\ss}e Anzahl an unabh\"angigen Konformationen
(etwa $10^{6}$) zu erzeugen. Wir betrachten die F\"alle $B=0$ und
$B=1$ auf einem Gitter $L=20$. Zur Vereinfachung setzen wir $J=1$.
Die Temperatur sollte im Bereich $T\in[0.5,4]$ variiert werden, zum
Beispiel $T=\{4,3,2.5,2.27,2.2,2.1,2,1.5,1,0.5\}$. 
\item Ermitteln Sie $\left\langle M\right\rangle $ und $\left\langle E\right\rangle $
als Funktion von $T$ aus diesen Konformationen und variieren Sie
$T$ dahingehend, dass Sie den Datenbereich $T>T_{c}$ und $T\rightarrow0$
abdecken. Tragen Sie ihre Ergebnisse f\"ur $B=0$ geeignet auf und vergleichen
Sie mit den verschiedenen Ergebnissen aus der Vorlesung - soweit m\"oglich.
Was beobachten Sie? 
\end{enumerate}

\subsubsection*{Importance Sampling}
\begin{enumerate}
\item Erweitern Sie das Programm um den Metropolis Algorithmus.
\item Wir betrachten zun\"achst den Fall $B=0$. Beginnen Sie mit einer zuf\"allig
erstellten Konformation und equilibrieren Sie das System mit Hilfe
von Metropolis. Ermitteln Sie die Magnetisierung $M$ als Funktion
der Simulationsdauer $t<500$ f\"ur 100 unabh\"angige Startkonformationen
und tragen Sie $\left\langle M(t)\right\rangle $ gegen $t$ auf.
Was beobachten Sie und welche Schlussfolgerungen ziehen Sie f\"ur die
Equilibrierung des Systems?
\item Ermitteln Sie $\left\langle M\right\rangle ,$ $\left\langle E\right\rangle $
und $C$ (aus den Fluktuationen von $E$) als Funktion von $T$ wobei
Sie nun f\"ur jede Temperatur zun\"achst equilibrieren und anschlie{\ss}end
eine ausreichende Zahl an Konformationen (mindestens 100) bei gegebener
Temperatur zur Auswertung ermitteln. Beginnen Sie mit hohen Temperaturen
und k\"uhlen Sie schrittweise weiter ab. Vergleichen Sie ihre Ergebnisse
mit den Resultaten von Simple Sampling und erkl\"aren Sie die Unterschiede.
\item Vergleichen Sie ebenfalls mit den L\"osungen aus der Vorlesung. 
\end{enumerate}

\subsubsection*{Externes Magnetfeld $B\protect\neq0$}

F\"uhren Sie die Aufgaben 2. und 3. der letzten Aufgabe f\"ur $B=1$ durch
und erl\"autern Sie Ihre Beobachtungen.
\end{document}
